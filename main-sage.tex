% sage_latex_guidelines.tex V1.20, 14 January 2017

\documentclass[Afour,sagev,times]{sagej}

\usepackage{moreverb,url}

\usepackage[colorlinks,bookmarksopen,bookmarksnumbered,citecolor=red,urlcolor=red]{hyperref}

\newcommand\BibTeX{{\rmfamily B\kern-.05em \textsc{i\kern-.025em b}\kern-.08em
T\kern-.1667em\lower.7ex\hbox{E}\kern-.125emX}}


\usepackage{lipsum} % for dummy text only. 


\usepackage{graphicx}
\graphicspath{ {./images/} }


\usepackage{caption}
\usepackage{xcolor}%



\usepackage{adjustbox}
\usepackage{float}
\usepackage{multirow}
\usepackage{subcaption}

\usepackage{hyperref}

\usepackage{booktabs}


\usepackage{todonotes}

\usepackage{amsmath,amssymb,amsfonts}%
\usepackage{amsthm}%

% see for more info
% https://tex.stackexchange.com/questions/229355/algorithm-algorithmic-algorithmicx-algorithm2e-algpseudocode-confused
%\usepackage{algorithm}%
%\usepackage[ruled,vlined]{algorithm2e}
%\usepackage{algorithmic}
\usepackage{algorithmicx}%
\usepackage{algpseudocode}%
\usepackage{listings}%



\def\volumeyear{2025}

\begin{document}

\runninghead{Özgür and SecondAuthor}

\input{title}

\author{Atilla Özgür\affilnum{1} and Second Author\affilnum{2}}

\affiliation{\affilnum{1} Company, Country\\
\affilnum{2} SAGE Publications Ltd, UK}

\corrauth{Atilla Özgür, Company
Department
, Country}

\email{email@gmail.com}


\input{abstract-begin-end}

\keywords{keyword1, Keyword2, Keyword3, Keyword4}

%%\pacs[JEL Classification]{D8, H51}

%%\pacs[MSC Classification]{35A01, 65L10, 65L12, 65L20, 65L70}





\maketitle


\section{Introduction}
\label{section-introduction}


introduction here

\begin{figure}[!htbp]
	\includegraphics{latex-logo.png}
	\caption{Example logo}
	\label{figure-example}
\end{figure}

\section{Related Works}
\label{section-related-works}


Özgür et al \cite{ozgur2021review} reviewed 90 articles.



\begin{table}[]
\caption{Related Works}
\label{table-related-works}
\begin{tabular}{@{}llll@{}}
Reference & Year & Feature 1 & Feature 2 \\
          &      &           &           \\
          &      &           &           \\
          &      &           &           \\
          &      &           &          
\end{tabular}
\end{table}

\section{Methods}
\label{section-methods}

\lipsum[10]


While using tables, put the table code in a different file like below example.
Here, table label and its filename should be same like below example.
You can use \url{https://www.tablesgenerator.com/} to easily generate latex tables from excel,csv and copy paste data.

\begin{table}[]
\caption{Latex Templates and Links}
\label{table-templates}
\begin{tabular}{@{}lll@{}}
\toprule
Template Name     & Journal Group             & Link \\ 
\midrule
Standard          & standard article template &      \\
IEEE Transactions & IEEE                      &      \\
Elsevier Journals & Elsevier                  &      \\
Springer Nature   & Springer                  &      \\ 
Sage    & Sage                  &      \\ 
\bottomrule
\end{tabular}
\end{table}

\lipsum[5]

\begin{figure}[!htbp]
	\includegraphics{latex-logo.png}
	\caption{Example logo}
	\label{figure-example}
\end{figure}



\section{Experimental Results}
\label{section-experimental-results}

\subsection{Datasets used in the study}
\label{sections-datasets}

datasets information







\section{Discussion}
\label{section-discussion}

discussion here

\lipsum[3]

\section{Conclusion}
\label{section-conclusion}

Conclusion here


\lipsum[3]




\section{Sage Journal Notes}


\subsection{End of paper special sections}
Depending on the requirements of the journal that you are submitting to,
there are macros defined to typeset various special sections.

The commands available are:
\begin{verbatim}
\begin{acks}
To typeset an
  "Acknowledgements" section.
\end{acks}
\end{verbatim}

\begin{verbatim}
\begin{biog}
To typeset an
  "Author biography" section.
\end{biog}
\end{verbatim}

\begin{verbatim}
\begin{biogs}
To typeset an
  "Author Biographies" section.
\end{biogs}
\end{verbatim}

%\newpage

\begin{verbatim}
\begin{dci}
To typeset a "Declaration of
  conflicting interests" section.
\end{dci}
\end{verbatim}

\begin{verbatim}
\begin{funding}
To typeset a "Funding" section.
\end{funding}
\end{verbatim}

\begin{verbatim}
\begin{sm}
To typeset a
  "Supplemental material" section.
\end{sm}
\end{verbatim}


%% add options to document class also
% sageh SAGE Harvard style (author-year)
% sagev SAGE Vancouver style (superscript numbers
% sageapa APA style (author-year)


%%Harvard (name/date)
%\bibliographystyle{SageH}
%%Vancouver (numbered)
\bibliographystyle{SageV}
\bibliography{./bib-files-and-documents/references,./bib-files-and-documents/helpful-references}






\end{document}
